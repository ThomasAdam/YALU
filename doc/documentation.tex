\documentclass[a4paper,11pt]{article}
\usepackage{a4-mancs}
\usepackage{booktabs}
\usepackage{longtable}
\usepackage{amssymb}

\newcommand{\yalu}{YetAnotherLevelUp}
\newcommand{\alu}{AnotherLevelUp}

\title{\yalu \\ \large{Documentation}}
\date{Version 0.1 Beta}

\begin{document}
	\maketitle

	\section{Introduction}
		\subsection{What \yalu\ is...}
			\yalu\ has been designed as a relatively powerful window-management-focused
			configuration for FVWM. It is intended to be a spiritual successor to \alu,
			(Which is itself a successor to AnotherLevel, itself a successor to The
			NextLevel).
			
			It is a complete rewrite of the entire code-base with the following goals:
			\begin{description}
				\item[Improve performance] \alu\ performance on networked computers is
					fairly poor (in part due to the large number of separate files it uses).
					It is also fairly bloated as a result of its age and scope.
				\item[Refocus the project scope] As mentioned above, \alu\ has a pretty massive
					code-base as it was originally intended to be a fairly comprehensive
					graphical environment. \yalu\ aims instead to provide a powerful
					graphical shell and floating window manager.
			\end{description}
			
			As a result of this, the resulting system is very different both
			architecturally and functionally. This may come as a relief to some and a
			disappointment to others\footnote{OK, just JTL...} but with a bit of time
			learning how to use it well you should hopefully find the experience
			pleasant.
		
		\subsection{What \yalu\ is \emph{not}...}
			\begin{itemize}
				\item \yalu\ is \emph{not} \alu. As described above it has different
				over-all goals.
				
				\item \yalu\ is not a desktop \emph{environment} like GNOME or KDE.
				These are giant projects spanning tens/hundreds of programs and aim to
				provide everything you could need to use a computer graphically. \yalu\
				is just a window manager and graphical shell.
			\end{itemize}
		
		\subsection{What next...}
			First off, install it obviously!
			
			Next, while I have included a quick-start, it is not (really) enough to
			get you using \yalu\ to anywhere near it's potential. You might like to
			skim over it just enough so that you'll be comfortable viewing this
			tutorial from within \yalu\ while trying things out.
			
			I'd recommend that you read through the Keyboard Shortcuts and Mouse
			Actions tables and then looking up anything that strikes your fancy. From
			there you're free to either plough on through the rest of the manual or to
			just play around in the menus.
			
			Have fun!
	
	\section{Installation}
		\subsection{Required Packages}
			\begin{itemize}
				\item FVWM (2.4 or later)
				\item Python (2.5 or later 2x release)
				\item dmenu
				\item Screen
				\item Zenity
				\item stalonetray
				\item xload
			\end{itemize}
		
		\subsection{For a Single User}
			\begin{enumerate}
				\item Copy the \texttt{.yalu} directory into your home folder
				\item Then simply instruct FVWM to read \texttt{\textasciitilde/.yalu/fvwmConfig}. For
					example you could:
					\begin{description}
						\item[Do it properly\texttrademark] Start FVWM with
							\texttt{fvwm2 -f "\textasciitilde/.yalu/fvwmConfig"}
						\item[Do it the easy way] Make a symlink between
							\texttt{\textasciitilde/.yalu/fvwmConfig} and
							\texttt{\textasciitilde/.fvwm/.fvwm2rc} and start FVWM normally.
					\end{description}
			\end{enumerate}
		\subsection{For a Multiple Users}
			Todo\footnote{
				While this is entirely possible with the current code base I am just far
				to lazy to ship a proper package which would make it trivial to do
				without splitting files up. See the \texttt{YALU} and \texttt{LocalYALU}
				environment variables in \texttt{fvwmConfig} for clues...
			}
	
	\section{Quick-Start}
		When you start \yalu\ it should look pretty much like any minimal window
		manager does when it starts. In other words boring. Don't sit there waiting
		for it to load because all you'll get is a wallpaper, a pager, tray and a
		mouse cursor and that should have loaded almost instantly.
		
		\subsection{Menus and Launching Programs}
		There are 5 important menus in \yalu\ and all of them can be accessed by
		clicking on the desktop. The two you probably care about right now are:
		\begin{description}
			\item[Left-click] Shows a launcher with a list of favourite programs.
			\item[Middle-click] Shows the \yalu\ configuration menu.
		\end{description}
		
		For all the programs in the launcher menu you'll notice they have one letter
		underlined, you can press this letter on the keyboard
		to select that item. You can also press Super+\emph{Letter} to launch that
		program from outside the menu.
		
		As well as the launcher, you can launch programs using dmenu. To use it,
		press Super\footnote{The `Windows' key on most
		keyboards}+Space\footnote{Sounds pretty awesome though, eh?}. You can now
		start typing a program name and dmenu will try and auto-complete it (along
		the top of the screen). Press enter to run that command or escape to cancel.
		
		\subsection{Window Buttons}
		Windows have similar buttons to those on mainstream environments. You can
		use them just the way you would normally. (Don't be fooled: if you
		read further on in the manual then you'll see they are actually far more
		powerful than they appear.)
		
		\subsection{Window Behaviour}
		By default, windows are not `raised' above others unless you click on their
		title bar (or press Alt+Space). You can `lower' a window by double clicking
		on its title bar (or by pressing Ctrl+Alt+Space). This may feel annoying at
		first but once you get used to it you'll find it very efficient (honest!).
		
		Windows are given keyboard focus whenever the mouse is over them. Again,
		this may seem annoying at first, especially if you are in the habit of
		knocking your mouse while typing. As \alu's original designer said: ``There
		is an easy fix for this: stop hitting your mouse.''
		
		\subsection{Desks and Pages}
		By default, you are given 1 `desktop' containing 6 `pages'. These pages are
		arranged in a 3$\times$2 grid. Moving between pages is effortless: simply
		move your mouse off the edge of the screen in the direction of the page you
		want to go to. You can drag windows like this and they will follow your
		mouse onto the next page too. You can also use Ctrl+Alt+\emph{Arrow} to
		move around from your keyboard. To move and take the currently focused
		window with you, hold down the Super key as well.
		
		\subsection{Really... You should read this manual...}
		While that is all you really need to know in order to be able to \emph{use}
		\yalu, you'll probably have a fairly miserable experience unless you also
		learn the less trivial features. Unfortunately it \emph{will} take you a
		reasonable amount of time and effort but just like learning how to use VIM
		or EMACS, you should easily make a return on your investment. At the very
		least have a look through the table of keyboard shortcuts and you'll get a
		taste for the sorts of features \yalu\ has to offer.
	
	\clearpage
	\newcommand{\subrule}{\cmidrule{2-3}}
	\section{Keyboard Shortcuts}
		\begin{longtable}{ l c l }
			\toprule
				Window Operations & Shift+Alt+F1 & Stick Window \\ \subrule
				                  & Alt+F2 & Iconify Window \\
				                  & Shift+Alt+F2 & Shade Window \\ \subrule
				                  & Alt+F3 & Maximize Window \\
				                  & Shift+Alt+F3 & Tall-Maximize Window \\
				                  & Super+Alt+F3 & Wide-Maximize Window \\ \subrule
				                  & Alt+F4 & Close Window \\
				                  & Super+Alt+F4 & Delete Window \\
				                  & Shift+Alt+F4 & Destroy Window \\ \subrule
				                  & Alt+F5 & Maximise Window into Space \\
				                  & Shift+Alt+F5 & Tall-Maximise Window into Space \\
				                  & Super+Alt+F5 & Wide-Maximise Window into Space \\ \subrule
				                  & Alt+F9 & Re-Place Window \\
				                  & Shift+Alt+F9 & Re-Place All Windows \\ \subrule
				                  & Alt+F10 & Re-Draw Window \\ \subrule
				                  & Alt+Space & Raise Window \\
				                  & Ctrl+Alt+Space & Lower Window \\ \subrule
				                  & Ctrl+Super+\emph{Arrow} & Fling Window in direction \\
				                  & Shift+Ctrl+Super+\emph{Arrow} & Throw Window in direction \\
			\midrule
				Icon Operations & Space \emph{or} Return & Open Focused Icon \\
				                & Ctrl+Return & Open Focused Icon into Space \\ \subrule
				                & Delete & Close Focused Icon \\
				                & Super+Delete & Delete Focused Icon \\
				                & Shift+Delete & Destroy Focused Icon \\ \subrule
				                & Shift+Alt+Space & Raise and Cycle Through Icons \\
				                & Shift+Ctrl+Alt+Space & Lower Icons \\
			\midrule
				Window Navigation & Super+\emph{Arrow}
				                  & Switch to application in direction \\
				                  & Alt+Tab & Cycle Through Window List \\
				                  & Shift+Alt+Tab & Cycle Backwards Through Window List \\
			\midrule
				Page Navigation & Ctrl+Alt+\emph{Arrow} & Switch to Page in direction \\
				                & Ctrl+Alt+Tab & Switch Previously Used Page \\ \subrule
				                & Ctrl+Super+Alt+\emph{Arrow}
				                & Move Window to Page in direction \\
				                & Ctrl++SuperAlt+Tab
				                & Move Window to Previously Used Page \\
			\midrule
				Desk Navigation & Shift+Ctrl+Alt+\emph{Arrow} & Switch to Desk in direction \\
				                & Shift+Ctrl+Alt+Tab & Switch Previously Used Desk \\ \subrule
				                & Shift+Ctrl+Super+Alt+\emph{Arrow}
				                & Move Window to Desk in direction \\
				                & Shift+Ctrl++SuperAlt+Tab
				                & Move Window to Previously Used Desk \\
			\midrule
				Pager and Tray & Super+Alt+Space & Raise Tray and Pager \\
				               & Ctrl+Super+Alt+Space & Lower Tray and Pager \\
			\midrule \clearpage
				Menus & Super+F1 & Launcher \\
				      & Super+F2 & \yalu\ Settings \\
				      & Super+F3 & Frequent/Recently Used \\
				      & Super+F4 & Running Program Output \\ \subrule
				      & Alt+F1 & Window Menu \\
			\midrule
				Applications & Super+T & Terminal \\
				             & Super+W & Web Browser \\
				             & Super+E & Editor \\ \subrule
				             & Super+\emph{Hotkey} & Other Programs in the Launcher \\ \subrule
				             & Super+Space & Run Application With dmenu \\
			\bottomrule
		\end{longtable}
		
		\subsection{Loose Conventions}
			In general, where possible (or sensible) these conventions are followed
			for keyboard shortcuts.
			\begin{longtable}{ c l }
				\toprule
					\dots+Space & A Quick-Access Function \\
					Shift+\dots & Inverts an Operation \\
					Ctrl+Alt & Refers to a Page Operation \\
					Super+\dots & Refers Programs or Windows \\
					\dots+\emph{F-key} & A Window Operation or Menu. \\
				\bottomrule
			\end{longtable}
			
			For example, Ctrl+Super+Alt+Right will move the current window to the page
			to the right because Ctrl+Alt means `Page Operation' and Super means act
			on a window. By contrast Ctrl+Alt+Space means `Lower Window' because the
			Space indicates that the shortcut is a quick-access function.
	
	\clearpage
	\section{Mouse Actions}
		\begin{longtable}{ l c l }
			\toprule
				Anywhere & Super+Drag-Click & Mouse Gesture \\
			\midrule
				Desktop & Left-Click & Show Launcher Menu \\
				        & Long\footnote{Long=Held Down}-Left-Click
				        & Show Running Program Output Menu \\
				        & Middle-Click & Show \yalu\ Configuration Menu \\
				        & Right-Click & Show Frequently/Recently Used Programs Menu \\
			\midrule
				Menus & Left-Click the Title & `Pin' the Menu \\
				      & Scroll-Wheel & Move Through Menu Items \\
			\midrule
				Over a Window & Alt+Drag-Left-Click & Move Window \\
				              & Alt+Drag-Right-Click & Resize Window \\
			\midrule
				Window Icon & Left-Click & Show Window Menu \\
			\midrule
				Window Title & Left-Click & Raise Window \\
				             & Double-Left-Click & Lower Window \\
				             & Drag-Left-Click & Move Window
				               \footnote{Didn't see that coming did you?} \\
				             & Long-Left-Click & Show Iconified Windows \\
				             & Middle-Click & Stick Window \\
				             & Right-Click & Re-Place Window \\
			\midrule
				Iconify Button & Left-Click & Iconify Window \\
				               & Right-Click & Shade Window \\
			\midrule
				Maximize Button & Left-Click & Maximize Window \\
				                & Long-Left-Click & Maximize Window into Space\\
				                & Middle-Click & Wide-Maximize \\
				                & Long-Middle-Click & Wide-Maximize Window into Space\\
				                & Right-Click & Tall-Maximize \\
				                & Long-Right-Click & Tall-Maximize Window into Space\\
			\midrule
				Close Button & Left-Click & Close Window \\
				             & Super+Left-Click & Delete Window \\
				             & Ctrl+Left-Click & Destroy Window \\
			\bottomrule
		\end{longtable}
	
	\newcommand{\ra}{$\rightarrow$}
	\clearpage
	\section{Menus}
		\subsection{General Features}
			\begin{itemize}
				\item Middle-click a menu title to `pin' it so it won't
					close\footnote{Note: Pinned menus will not update, for example if you
					pin the recently used programs menu it will not change when you run new
					programs}. This is useful if you are going to be changing a lot of
					settings or want to create a quick launcher menu.
				\item Use the scroll wheel on your mouse to move through menu entries.
				\item Press the under-lined letter on a menu item to jump to that one.
					If there are no other items sharing the same letter then it will act
					as if you clicked it.
			\end{itemize}
		
		\subsection{Launcher}
			By default it contains a Terminal, Web Browser and Editor. You can pick
			the programs that these functions launch in \yalu\ Configuration Menu \ra
			Programs. Each entry is given an icon based on the first word of the
			command it represents (usually the program name).
			
			For all entries in this menu the hotkey (indicated by an underlined
			letter) is globally accessible by holding Super and that key.
			
			Some entries may also have an associated mouse gesture. These can be used
			by holding down the super key and drawing it anywhere on the screen. The
			following gestures for the default menu items are provided:
			\begin{longtable}{l l}
				Terminal & Line from Left-to-Right \\
				Web Browser & Line going Up and then Right (in a right-angle) \\
				Editor & Line going Down and then Right (in a right-angle) \\
			\end{longtable}
			
			\subsubsection{Editing the Launcher}
				The menu is generated from a file called \texttt{menu} in the \yalu\
				directory. You can quickly open it from the \yalu\ Configuration Menu
				\ra Programs \ra Edit Launcher.
				
				The file contains menu entries on separate lines with the label and
				command separated by a tab. If the label and program name are the same
				then you can omit the program name entirely. To add a spacer in the menu
				simply leave an empty line.
				
				By default the first letter of the label is used for the hotkey. You can
				specify another letter by putting an ampersand (\&) in front of the
				letter you wish to use. If more than one command has the same hotkey
				then when you press Super+\emph{Hotkey} a menu will pop up allowing you
				to pick between the possible matches. The next letter in
				the command's label is used as a hotkey in this menu. This means that if
				you have both xclock and xlock bound to the hotkey `x' you could press
				Super+X then C to run xclock.
				
				To assign a mouse gesture to a menu item then you simply place a
				gesture-code inside brackets ( \{ and \} ) at the end of the
				label.\footnote{Note: If you assign the same gesture to more than one
				command then the last command to be assigned the gesture will over-ride
				the others} The gesture-codes are a list of numbers indicating the shape
				to be drawn based on the grid below:
				\begin{longtable}{c c c}
					1 & 2 & 3 \\
					4 & 5 & 6 \\
					7 & 8 & 9 \\
				\end{longtable}
				For example, an `N' shaped gesture-code would be \{7415963\}.
				
			\subsubsection{Example Launcher \texttt{menu} File}
				\begin{verbatim}
					GUI &File Browser {7412369}       nautilus --no-desktop
					
					x&clock {3214789}
					x&lock
					
					VLC                               vlc
					xine
					Totem                             totem
					Play All Music                    vlc ~/Music/*
				\end{verbatim}
		
		\subsection{\yalu\ Configuration Menu}
			Lalala
	
	\section{Options}
		\subsection{Focus Modes}
			\yalu\ Configuration Menu

\end{document}
